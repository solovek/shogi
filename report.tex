% author: Matheus Figueiredo

\documentclass[12pt]{article}

\usepackage{standalone}
\usepackage{sbc-template}
\usepackage[utf8]{inputenc}
\usepackage[brazil]{babel}
\usepackage{tikz}
\usetikzlibrary {automata, positioning, arrows}

\title{shogi}
\author{Matheus Figueiredo}
\address{Curso de Bacharelado em Ciência da Computação\\
  Universidade Federal de Pelotas (UFPel)
  \email{matheusfschmalfuss@inf.ufpel.edu.br}
}

\begin{document}
\maketitle

%\begin{abstract}
%\end{abstract}

\begin{resumo}
  O presente artigo é uma descrição de uma implementação em c++ do clássico jogo shogi, utilizando o protocolo TCP para passagem de mensagens e um modelo peer to peer.
\end{resumo}

\section{Introduction}
Shogi é uma das variações do jogo chaturanga, do qual o xadrez também é uma variação. É jogado em um tabuleiro 9x9 e notável pelas peças capturadas serem realmente capturadas, no sentido de que o jogador que as capturou pode num momento futuro botar as peças novamente no tabuleiro.\\

\section{Protocolo}
\begin{center}
  \includestandalone{state-machine}
\end{center}

\section{Mensagens}
As mensagens de movimento tomam a forma \{mP(yx)?[-*x](yx)+?\} onde 'm' faz alusão a "movimento", P denota a peça a ser movida, o primeiro (yx) as coordenadas de origem (opcional), seguido do tipo de movimento, as coordenadas de destino e (caso haja promoção da peça) um '+'.\\
      
\begin{table}[h!]
  \begin{center}
    \begin{tabular}{l|c}
      \textbf{Movimento} & \textbf{Significado}\\
      \hline
      -   & Move peça do tabuleiro pra posição indicada.\\
      $*$ & Move peça da mão pro tabuleiro na posição indicada.\\
      x   & Captura peça do adversário na posição indicada.\\
    \end{tabular}
  \end{center}
\end{table}

\begin{table}[h!]
  \begin{center}
    \begin{tabular}{l|c}
      \textbf{Mensagem} & \textbf{Significado}\\
      \hline
      f  & Anuncia pra rede que quer jogar.\\
      rq & Envia um match request prum jogador que tenha se anunciado.\\
      b[f][BOARD] & Se f, BOARD contém uma lista de peças, caso contrário utiliza-se a forma padrão.\\
      rs & Desiste do jogo.\\
      rm & Pede rematch (ou aceita o rematch, quando enviado em resposta a outro rm).\\
    \end{tabular}
  \end{center}
\end{table}

      
\bibliographystyle{sbc}
\bibliography{sbc-template}

\end{document}
