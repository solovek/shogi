% author: Matheus Figueiredo

\documentclass[12pt]{article}

\usepackage{standalone}
\usepackage{sbc-template}
\usepackage[utf8]{inputenc}
\usepackage[brazil]{babel}
\usepackage{tikz}
\usetikzlibrary {automata, positioning, arrows}

\title{shogi}
\author{Matheus Figueiredo}
\address{Curso de Bacharelado em Ciência da Computação\\
  Universidade Federal de Pelotas (UFPel)
  \email{matheusfschmalfuss@inf.ufpel.edu.br}
}

\begin{document}
\maketitle

\begin{abstract}
\end{abstract}

\begin{resumo}
  escreva-me
\end{resumo}

\newpage
\section{Protocolo}
\includestandalone{state-machine}

\section{Mensagens}
\begin{itemize}
\item b[board]: host envia posição das peças ao outro jogador;
\item mv[yx][yx]: descreve ao outro jogador que deve checar a sua validade o movimento desejado;
\item dp[p][yx]: jogador põe uma peça "p" no tabuleiro;
\item invalid: informa ao outro jogador que a última jogada recebida é inválida;
\item connect: jogador pede pra se conectar ao host;
\item resign: jogador concede a vitória (também usado quando há cheque mate).
\end{itemize}

\bibliographystyle{sbc}
\bibliography{sbc-template}

\end{document}
